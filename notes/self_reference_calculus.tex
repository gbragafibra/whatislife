\documentclass[a4paper,12pt,twoside,leqno]{article}
\usepackage[marginratio={5:5, 5:5}, textwidth=150mm, ]{geometry}


\pagenumbering{gobble}
\usepackage{amssymb}
\usepackage{xcolor}
\usepackage{amsmath}
\usepackage{hyperref}
\usepackage{tikz}
\usetikzlibrary{quotes}
\usetikzlibrary{automata, arrows.meta, positioning}
\usetikzlibrary{calc}

\title{A primer on calculus of indications}
\author{Gonçalo Braga}

\newcommand{\marked}[1]{%
  \tikz[baseline=(base)]{
    \node[inner sep=0pt, outer sep=0pt] (base) {\strut#1};
    % Draw the top and right lines of the corner
    \draw[thick] 
      ($(base.north west) + (-0.4em, 0.2em)$) -- ($(base.north east) + (0.2em, 0.2em)$) -- 
      ($(base.north east) + (0.2em, -0.2em)$) -- ($(base.south east) + (0.2em, 0.4em)$);
  }%
}


\begin{document}
\maketitle
\abstract{A primer on calculus of indications, specifically building towards an understanding of Varela's extended calculus. Very informal summary.}
\subsection*{Calculus of indications}
The marked state is presented by $\marked{}$\,, and the unmarked state is presented by the blank $\quad$. The two-fold meaning of making a distinction with $\marked{}$ allows for a calculus without the common operand/operator distinction.
\subsection*{Initials}
\begin{equation}\label{1}
\marked{}\, \marked{} = \marked{}
\end{equation}
\begin{equation}\label{2}
\marked{\marked{}} = 
\end{equation}
\subsection*{Primary arithmetic}
Consequences from initials \ref{1} and \ref{2}, with the example:
$$
\marked{\marked{}\,\marked{}}\,\marked{\marked{\marked{}}}\,\marked{} = \marked{}\,\marked{} = \marked{}
$$
As for the central axioms of this primary arithmetic, we have:
\begin{equation}
\marked{\marked{p}p} = 
\end{equation}
and 
\begin{equation}
\marked{\marked{pr}\,\marked{qr}} = \marked{\marked{p}\,\marked{q}}r
\end{equation}
\subsection*{Re-entry in the primary algebra}
In order to fix situations of infinite regress like the following
$$
\marked{\marked{\marked{\dots}}}
$$
one can have re-entry in the following manner, much as with the Y-combinator in the $\lambda$-calculus:
\begin{equation}
f = \marked{f}
\end{equation}
where one gets a fixed and finite expression, for what would otherwise be evaluated onto:
\begin{equation}
f = \marked{f} = \marked{\marked{f}} = \marked{\marked{\marked{\dots}}}
\end{equation}
\end{document}