\documentclass[a4paper,12pt,twoside,leqno]{article}
\usepackage[marginratio={5:5, 5:5}, textwidth=150mm, ]{geometry}


\pagenumbering{gobble}
\usepackage{amssymb}
\usepackage{xcolor}
\usepackage{amsmath}
\usepackage{pdfrender}
\usepackage{tikz}
\usetikzlibrary{quotes}
\usetikzlibrary{automata, arrows.meta, positioning}
\pdfrender{StrokeColor=black,TextRenderingMode=2,LineWidth= 0.5 pt}
\newtheorem{question}{Question}
\newtheorem{note}{Quick Note}
\title{ORGANISMS AREN'T, THEY HAPPEN}
\author{\small{GONÇALO BRAGA}}
\date{}

\begin{document}
\maketitle
\abstract{This is merely to keep track of where, and how I'm thinking about the problem, in a very informal manner. A sort of log if you will.}
\subsection*{Introductory statement}
Organisms and life (the process) are actually things that are very illusive. We usually go on to describe proxies of them, as in genetics, molecular biology, and overall life-sciences, but organizational principles and a description of what organismal organization is, is missing. There are have been a miriad of attempts at doing so, and a common theme is that of relations. An organism is in virtue of the relations between its constituents. What matters are the relations, and not the constituents themselves. That's why some people (Varela, etc), refer to what we see as an physical instantiation of life. If there would be other substrates that nonetheless show the same relations between themselves, such system would be characterized as an organism.\\
Even more than this central theme, is one derived from it: self-reference or impredicativity. It means that such systems are self-determined, and can be illustrated very simply, as by what Rosen (presumably one of the first to put the argument into completely relational terms with category theory) pointed out. If we take $f$ as metabolism, we have
\begin{equation}
f(f) = f
\end{equation}
Here $f$ is serving as function, argument and result. Mathematics as a whole tries to push away such infinite regress, and such characteristic is the bane of its existence (presumably only under classical logic; such that three-value logics might be able to go around this problem, or even just using constructive logic). A bunch of paradoxes: Liar paradox (Epimenides'), Halting problem, Russel's paradox, etc, derive from this. One needs to understand, though, that under this infinite regress $f(f(f(f(...))) = f$, the ambiguity in understanding where such object is serving under each role (function, argument, etc), is precisely why semantics can't be completely reduced to syntax, and this shouldn't be avoided. It should be actually explored in a full manner, much like the way that the real domain was expanded into the complex domain by addition of another object $i = \sqrt{-1}$, by having the contradiction $x^2 = -1$. Here $x$ would need to be both positive and negative for it to follow. This is the type of exercise which is taken by three-value logics, and more specifically with Spencer-Brown's calculus of indications, which was later extended by Francisco Varela. 
\subsection*{On modelling impredicativity and self-reference}
Dynamical systems theory (be it non-linear or not) as far as I'm aware only deals with the evolution of state variables in a deterministic or stochastic manner, according usually to a set of ODEs/PDEs. We might also have more complex behaviour by having some of these being coupled to each other. Over the corresponding phase-space, there can be analysis of the stability of the fixed-points, if they exist, given small pertubations. However what's missing is the notion of blending both operand and operator. In this case, we would have a reflexive space under which any object or state also acts as a transformation. We want the corresponding state-evolution to also affect meta-dynamics (imagining here a changing set of PDEs). Even more important is the stability of organization, that is the stability of relations between constituents. Modelling organismal organization, I assume, needs these types of approaches. On that matter, $\lambda$-calculus allows for functions to be both operands and operators. Some concepts in category theory are also useful. Furthermore, there's Spencer-Brown's calculus of indications which does also have some interesting concepts, namely those extended by Francisco Varela, Louie Kauffman, etc.\\
So in essence, one is looking for a way to express fixed-points over organization, that is, over relationships between processes of the system, and not necessarily fixed-points regarding state-variables. There would be various ways to have the same fixed-point organizationally that nonetheless wouldn't correspond to fixed-points if one is looking at state-variables. This is the problem of modelling impredicativity.
\subsection*{Similar approaches?}
Regarding Rosen's approach with category theory, and Varela's with his extension of the calculus of indications, it largely seems (atleast at surface) that these approaches are very similar. Particularly, in the similarity of Varela's third state of re-entry, or the autonomous state, with the concept of an endomorphism. An endomorphism describes an object mapping to itself. The same could be characterized for the autonomous state. Rosen constructed his (M-R) system mapping to avoid the infinite regress of organismal organization (or to express it in a finite form). The same could be said for Varela's approach, with a three-value logic. One can wonder what would be, if they had actually been aware of each others work. There needs to be a better way of conceptualizing fixed-points over organization, under which there's a reflexive domain. That is a domain, for which the objects also act as transformations. 
\subsection*{The ouroboros dilemma}
The ouroboros equation is given by $f(f) = f$ as previously mentioned. Although having the correct framework to find non-trivial solutions for such is difficult, I would argue the most difficult part is the association between such object and processes in the natural world. Rosen specifically chooses $f$ as metabolism (as in cellular metabolism), and has other objects and mappings to close his (M, R)-system under which every efficient cause is corroborated by a material one. Such objects and mappings lead to closure to efficient causation, and associated organizational invariance. My question: would this abstract model capture every instantion of an organism? My contention comes from the "repair" or regeneration part, under which if not the actual components of a set, atleast the corresponding "classes" of such set would need to be regenerated. Why? Why can't we have run-away-like systems for which such sets aren't regenerated if not fully, atleast partially? This comes, I guess from my limited capacity, but of not understanding associated concepts of closure. Even closure of constraints, which is a similar concept, usually could also be said to happen in other dynamical systems which are clearly not alive (weather patterns, fire, etc). So in this capacity one would either affirm that such closure isn't maintained for a long enough time, or it has a small number of associated constraints (which by default would be very unlikely to be regulatory). Which gets us to consider again the obvious: There's a huge amount of evolutionary momentum. Why are we taking some concepts and events for granted? Even more specifically why are we confusing the evolution of such systems, with their ontology. One assumes, typically, that these are related, but this is a very Newtonian assumption. This regeneration of sets, which we typically call metabolism, seems to me to be an assumption. It's clearly easier to maintain closure due to it, but I don't think it's a necessary condition. The same goes for other processes (such as replication, which would fall under regeneration of sets, or a typical boundary), which could be better viewed as adaptations to compensations, in order to maintain a specific organization. This organization, the organismal one, which allows dialectical processes (ones that are co-dependent existentially on each other) to emerge such that they and their relations evolve so as to maintain this very specific organization. In this way this organization $\Phi$ is what could be lent to be a fixed-point, and have the associated ouroboros equation $\Phi (\Phi) = \Phi$. The problem is defining such organization in terms of observables that are amenable to inspection (in a practical manner) in both organisms and other dynamical systems.
\subsection*{What I don't understand about (M, R)-systems}
I either completely miss what Rosen puts forward, or I actually think they are too specific.\\
An organizationally invariant (M, R)-system is viewed through the respective mappings:
$$
A \xrightarrow{f} B \xrightarrow{\Phi} H(A, B) \xrightarrow{\beta} H(B, H(A, B))
$$
for which $f(a) = b$, $\Phi(b) = f$ and $\beta(f) = \Phi$, such that it's closed to efficient causation. The first mapping $f$ refers to metabolism and the corresponding transformation of reactants to products under the action of catalysts, $f: A \rightarrow B$. The second mapping $\Phi$ is associated to the "repair" system, which regenerates the corresponding catalysts which allow the morphism $f$. The corresponding morphism is $\Phi: B \rightarrow H(A, B)$ as it is repairing from all possible sets of metabolisms ($H(A, B)$). Following that, such repair system needs to be generated, and needs to be generated from within. As such we have a third mapping $H(A, B) \xrightarrow{\beta} H(B, H(A, B))$. As such, the only possible metabolism (the current one) is used so as to regenerate all possible metabolisms, that is, $\beta(f) = \Phi$. In order to avoid the infinite regress, Rosen allows $\Phi(b) = f$ to have only one solution. In this way, $\beta$ needs no further constraints, and is taken as an inverse of the morphism $\Phi(b) = f$. This is a pretty big constraint, as noted by Soto-Andrade et al. (2011). If we interpret this system, we have a certain metabolism $f$, whose catalysts (which are eventually degraded) are eventually regenerated by $\Phi$, and which repair system is then generated from within with $\beta$. However, the only metabolism that is allowed to be used under the morphism $\beta$ is $f$ and not any other metabolism in the set of all possible metabolisms $H(A, B)$.\\
I don't see how this isn't to restrictive, and I guess Soto-Andrade et al. (2011) develop on that. My problem resides, again, on why regenerate a set, multiple sets, or even a category of sets? Why? I legitimately don't understand why such regeneration isn't an assumption, and a big one at that, for the nature of organismal organization? Mistaking the physiology for the ontology?
\end{document}
