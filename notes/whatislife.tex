\documentclass[a4paper,12pt,twoside,leqno]{article}
\usepackage[marginratio={5:5, 5:5}, textwidth=150mm, ]{geometry}


\pagenumbering{gobble}
\usepackage{amssymb}
\usepackage{xcolor}
\usepackage{amsmath}
\usepackage{pdfrender}
\usepackage{tikz}
\usetikzlibrary{quotes}
\usetikzlibrary{automata, arrows.meta, positioning}
\pdfrender{StrokeColor=black,TextRenderingMode=2,LineWidth= 0.5 pt}
\newtheorem{question}{Question}
\newtheorem{note}{Quick Note}
\title{ORGANISMS AREN'T, THEY HAPPEN}
\author{\small{GONÇALO BRAGA}}
\date{}

\begin{document}
\maketitle
\abstract{This is merely to keep track of where, and how I'm thinking about the problem, in a very informal manner. A sort of log if you will.}
\subsection*{Introductory statement}
Organisms and life (the process) are actually things that are very illusive. We usually go on to describe proxies of them, as in genetics, molecular biology, and overall life-sciences, but organizational principles and a description of what organismal organization is, is missing. There are have been a miriad of attempts at doing so, and a common theme is that of relations. An organism is in virtue of the relations between its constituents. What matters are the relations, and not the constituents themselves. That's why some people (Varela, etc), refer to what we see as an physical instantiation of life. If there would be other substrates that nonetheless show the same relations between themselves, such system would be characterized as an organism.\\
Even more than this central theme, is one derived from it: self-reference or impredicativity. It means that such systems are self-determined, and can be illustrated very simply, as by what Rosen (presumably one of the first to put the argument into completely relational terms with category theory) pointed out. If we take $f$ as metabolism, we have
\begin{equation}
f(f) = f
\end{equation}
Here $f$ is serving as function, argument and result. Mathematics as a whole tries to push away such infinite regress, and such characteristic is the bane of its existence (presumably only under classical logic; such that three-value logics might be able to go around this problem, or even just using constructive logic). A bunch of paradoxes: Liar paradox (Epimenides'), Halting problem, Russel's paradox, etc, derive from this. One needs to understand, though, that under this infinite regress $f(f(f(f(...))) = f$, the ambiguity in understanding where such object is serving under each role (function, argument, etc), is precisely why semantics can't be completely reduced to syntax, and this shouldn't be avoided. It should be actually explored in a full manner, much like the way that the real domain was expanded into the complex domain by addition of another object $i = \sqrt{-1}$, by having the contradiction $x^2 = -1$. Here $x$ would need to be both positive and negative for it to follow. This is the type of exercise which is taken by three-value logics, and more specifically with Spencer-Brown's calculus of indications, which was later extended by Francisco Varela. 
\subsection*{On modelling impredicativity and self-reference}
Dynamical systems theory (be it non-linear or not) as far as I'm aware only deals with the evolution of state variables in a deterministic or stochastic manner, according usually to a set of ODEs/PDEs. We might also have more complex behaviour by having some of these being coupled to each other. Over the corresponding phase-space, there can be analysis of the stability of the fixed-points, if they exist, given small pertubations. However what's missing is the notion of blending both operand and operator. In this case, we would have a reflexive space under which any object or state also acts as a transformation. We want the corresponding state-evolution to also affect meta-dynamics (imagining here a changing set of PDEs). Even more important is the stability of organization, that is the stability of relations between constituents. Modelling organismal organization, I assume, needs these types of approaches. On that matter, $\lambda$-calculus allows for functions to be both operands and operators. Some concepts in category theory are also useful. Furthermore, there's Spencer-Brown's calculus of indications which does also have some interesting concepts, namely those extended by Francisco Varela, Louie Kauffman, etc.\\
So in essence, one is looking for a way to express fixed-points over organization, that is, over relationships between processes of the system, and not necessarily fixed-points regarding state-variables. There would be various ways to have the same fixed-point organizationally that nonetheless wouldn't correspond to fixed-points if one is looking at state-variables. This is the problem of modelling impredicativity.
\subsection*{Similar approaches?}
Regarding Rosen's approach with category theory, and Varela's with his extension of the calculus of indications, it largely seems (atleast at surface) that these approaches are very similar. Particularly, in the similarity of Varela's third state of re-entry, or the autonomous state, with the concept of an endomorphism. An endomorphism describes an object mapping to itself. The same could be characterized for the autonomous state. Rosen constructed his (M-R) system mapping to avoid the infinite regress of organismal organization (or to express it in a finite form). The same could be said for Varela's approach, with a three-value logic. One can wonder what would be, if they had actually been aware of each others work. There needs to be a better way of conceptualizing fixed-points over organization, under which there's a reflexive domain. That is a domain, for which the objects also act as transformations. 
\subsection*{The ouroboros dilemma}
The ouroboros equation is given by $f(f) = f$ as previously mentioned. Although having the correct framework to find non-trivial solutions for such is difficult, I would argue the most difficult part is the association between such object and processes in the natural world. Rosen specifically chooses $f$ as metabolism (as in cellular metabolism), and has other objects and mappings to close his (M, R)-system under which every efficient cause is corroborated by a material one. Such objects and mappings lead to closure to efficient causation, and associated organizational invariance. My question: would this abstract model capture every instantion of an organism? My contention comes from the "repair" or regeneration part, under which if not the actual components of a set, atleast the corresponding "classes" of such set would need to be regenerated. Why? Why can't we have run-away-like systems for which such sets aren't regenerated if not fully, atleast partially? This comes, I guess from my limited capacity, but of not understanding associated concepts of closure. Even closure of constraints, which is a similar concept, usually could also be said to happen in other dynamical systems which are clearly not alive (weather patterns, fire, etc). So in this capacity one would either affirm that such closure isn't maintained for a long enough time, or it has a small number of associated constraints (which by default would be very unlikely to be regulatory). Which gets us to consider again the obvious: There's a huge amount of evolutionary momentum. Why are we taking some concepts and events for granted? Even more specifically why are we confusing the evolution of such systems, with their ontology. One assumes, typically, that these are related, but this is a very Newtonian assumption. This regeneration of sets, which we typically call metabolism, seems to me to be an assumption. It's clearly easier to maintain closure due to it, but I don't think it's a necessary condition. The same goes for other processes (such as replication, which would fall under regeneration of sets, or a typical boundary), which could be better viewed as adaptations to compensations, in order to maintain a specific organization. This organization, the organismal one, which allows dialectical processes (ones that are co-dependent existentially on each other) to emerge such that they and their relations evolve so as to maintain this very specific organization. In this way this organization $\Phi$ is what could be lent to be a fixed-point, and have the associated ouroboros equation $\Phi (\Phi) = \Phi$. The problem is defining such organization in terms of observables that are amenable to inspection (in a practical manner) in both organisms and other dynamical systems.
\subsection*{What I don't understand about (M, R)-systems}
I either completely miss what Rosen puts forward, or I actually think they are too specific.\\
An organizationally invariant (M, R)-system is viewed through the respective mappings:
$$
A \xrightarrow{f} B \xrightarrow{\Phi} H(A, B) \xrightarrow{\beta} H(B, H(A, B))
$$
for which $f(a) = b$, $\Phi(b) = f$ and $\beta(f) = \Phi$, such that it's closed to efficient causation. The first mapping $f$ refers to metabolism and the corresponding transformation of reactants to products under the action of catalysts, $f: A \rightarrow B$. The second mapping $\Phi$ is associated to the "repair" system, which regenerates the corresponding catalysts which allow the morphism $f$. The corresponding morphism is $\Phi: B \rightarrow H(A, B)$ as it is repairing from all possible sets of metabolisms ($H(A, B)$). Following that, such repair system needs to be generated, and needs to be generated from within. As such we have a third mapping $H(A, B) \xrightarrow{\beta} H(B, H(A, B))$. As such, the only possible metabolism (the current one) is used so as to regenerate all possible metabolisms, that is, $\beta(f) = \Phi$. In order to avoid the infinite regress, Rosen allows $\Phi(b) = f$ to have only one solution. In this way, $\beta$ needs no further constraints, and is taken as an inverse of the morphism $\Phi(b) = f$. This is a pretty big constraint, as noted by Soto-Andrade et al. (2011). If we interpret this system, we have a certain metabolism $f$, whose catalysts (which are eventually degraded) are eventually regenerated by $\Phi$, and which repair system is then generated from within with $\beta$. However, the only metabolism that is allowed to be used under the morphism $\beta$ is $f$ and not any other metabolism in the set of all possible metabolisms $H(A, B)$.\\
I don't see how this isn't to restrictive, and I guess Soto-Andrade et al. (2011) develop on that. My problem resides, again, on why regenerate a set, multiple sets, or even a category of sets? Why? I legitimately don't understand why such regeneration isn't an assumption, and a big one at that, for the nature of organismal organization? Mistaking the ontology for the physiology?
\subsection*{On material impredicativity}
If we take into account material impredicativity, and with it Turing machines, the reading head and corresponding rules that govern it, would need to co-create each other, changing accordingly. In this analogy, an organism would be a Turing machine that, additionally to the prior description, also builds its own reading head ("from within"). This steems with self-reference. Majority of the diagonal arguments used to solve "paradoxes" (Cantor's, Russell's, Gödel's, Turing's halting problem, etc) seem to be using classical logic, which allows proof by contradiction, which wouldn't be allowed in constructive logic given that it does not assume the law of the excluded middle. There needs to be a better way to talk about self-reference and actually take these "paradoxes" for what they are: Non-existent cartesian dualities. In trying to understand organismal organization, what "shouts at us" immediately is self-reference. There's no way to go around this. These systems define themselves from within. The problem resides in formalizing, if possible, such ontology and corresponding evolution of such systems.  
\subsection*{What gives?}
There's a peculiar similarity between these "problems", that condense into: semantics can't be fully reduced to syntax, as the consistency of these systems can't be fully addressed from within (Tarski's, Cantor's, Gödel's, etc), and the approach that would be needed to address the self-referential organization in organisms. I go back to a quote from Varela's paper on "A calculus for self-reference":
\begin{quotation}
It appears as if different, successively larger levels are connected and intercross at the point where the constituents of the new lower level refer to themselves, where antinomic forms appear, and time sets in.
\end{quotation}
Self-reference or impredicativity is the name of the game.
\subsection*{(M, R)-system in type-free $\lambda$-calculus}
As given in Mossio et al. (2009), although with some contentions with assumptions as discussed in Cárdenas et al.(2010), Rosen's canonical (M, R)-system can be put into untyped $\lambda$-calculus in the following manner. The corresponding morphisms would be, given that we associate $\beta = B$:
$$
(fA) = B
$$
$$
(\Phi B) = f
$$
and
$$
(Bf) = \Phi
$$
Replacing $B$ and then $\Phi$ we get:
$$
f = ((fA)f)(fA)
$$
We can now introduce $G = \lambda x.((xA)x)xA$ and the $Y$ combinator as $Y = \lambda y.(\lambda x.y(xx))(\lambda x.y(xx))$, from which :
$$
G(YG) = YG
$$
follows. Introducing $f = YG$, we then state:
$$
f = Gf = ((fA)f)(fA)
$$
given that we'll have:
$$
B = ((YG)A) = YGA
$$
and
$$
\Phi = (B(YG)) = (YGA)(YG)
$$
As such, we have $f$, $\Phi$ and $B$ all defined in terms of $A$ and the $Y$ combinator. There are some arguments in Mossio et al. (2009) asserting that Rosen's (M, R)-system is not really impredicative, and even if it were, such condition wouldn't preclude it from being computable, as in simulation sake. Cárdenas et al. (2010) say that the computability of such diagram isn't valid given that the association $\beta = B$ is not valid, and that the strenght of the proof given by Mossio et al. (2009) is weaker than Rosen's version which is based on Turing's definition of computability. 
\subsection*{Is the concept of closure a necessity, a sufficiency, both or neither?}
When talking about closure of constraints or organizational closure, usually it is said that closure is achieved when each of the constraints is generating atleast another constraint in the system, such that the system achieves closure from within. There's another problem though, which is: how should one discern if some thing or some process is a constraint. Mossio and Montévil say that such constraint needs to be conserved (or have symmetry) at a certain time scale $\tau$. If this is a good way to go about it, I'm not sure, as literally everything will be considered a constraint at a certain scale under this definition. More importantly, I believe this suffers from the same problem than Rosen's approach does. When we talk about identifying processes or components to be in a certain set or category of sets, we are inviting a problem. Every component or process, apart from being mainly "abstracted" into a single set, is also going to influence (and be explicitely) in every other set. The system needs to be considered as a whole. Decomposition like this tries to mask away the problem. How are we going to prove that a cell might have closure? With our abstract models? Assuming that the modularity of function we see in biochemical circuits is really what's going on in cells? Talking about $N$ state variables and the corresponding phase-spaces, and then taking the very strong assumption "if all else equal" then our analysis of such space, along with attractors, fixed-points and trajectories is valid? We might be fooling ourselves with this approach even. Take a relationship between $N$ state variables and attribute to them a certain behaviour, and assume the rest of the variables that could potentially be considered to describe the system don't influence such space? Do we go about computing sensitivities between state variables ad-infinitum? This is not being lazy enough. And potentially missing a big point.\\
One of the problems that seem to be associated to closure, is that other dynamical systems also have it. There are arguments that presumably the complexity of the constraints isn't as high as with the one in organisms (emergence of regulative constraints, which after pertubations control the constitutive constraints, but which need an explanation for their emergence), or that perhaps closure isn't as achieved for enough time. I think something that might make it clear, is that the organization of such systems is different. In organisms, if we take $\Phi$ to be an organizational closure function (which would be a function of the underlying processes and relations between them), we have:
$$
\Phi (\Phi) = \Phi
$$
which in other words means that it is a fixed-point. It can be considered as autonomous organizational closure. In other dynamical systems, we would have:
$$
\Phi (\Phi) \neq \Phi
$$
In organisms, any type of adaptation (e.g. metabolism, replication, aging, etc) is to be taken as a compensation for a pertubation (even for $n$-th order pertubations which would emerge from within the system) in order to conserve $\Phi$. Although it presumably gets more complicated as scale is increased. This is a very specific type of organization. It is the one under which the dynamics evolve so as to conserve it. It is inherently self-referential. The problem resides in explaining its emergence.\\
From such equality and inequality seen prior, we can state that in organisms we have an infinite regress of the type $\Phi(\Phi(...)) = \Phi$, and in other dynamical systems this isn't the case. After a certain point the organization function $\Phi$ is no longer a "fixed-point". Which it would never be from the start, as fixed-points imply infinite regress. This is I presume the main difference. Even if one could point to organizational closure in both organisms and other dynamical systems, the type of organization seen in organisms is a fixed-point. We should be wary of the huge amount of evolutionary momentum that there is, which leads to confuse the ontology of the organisms with their evolution or physiology. These are presumably very different. Other results obviously associated to organisms, are those of Prigogine and other authors, associated with non-equilibrium and stochastic thermodynamics. Again, organisms are far-from-equilibrium systems, they are thermodynamically open systems, but organizationally closed ones. \\
At what point (how?) does $\Phi$ become a fixed-point. Is closure (of constraints) enough to describe such organization? Is it even necessary? This organization is defined in terms of its capability to conserve itself. The processes and their relations, which "compose" (using the word very loosely) the system, evolve in a manner so as to conserve $\Phi$. How shall $\Phi$ be described and defined. Is it equivalent to closure (of constraints)? Wouldn't a run-away system (which never returns to the same initial set (or classes) of components) be capable of such, eventhough it (presumably) doesn't show closure of constraints? It's as if organisms are a universe of their own, when it comes to their organization or operation, eventhough clearly thermodynamically open.
\subsection*{Organism as a universe of its own}
I presume what I mean by organization $\Phi$ of a system, such that $\Phi (\Phi) = \Phi$ is more, or atleast different from the concept of closure of constraints. As such, an illustration might provide more insight, given that the self-reference can't be taken head on. If we go back to some though by Ernst Mach, regarding inertia (which by the equivalence principle would be equivalent to gravitation) being originated by all masses in the cosmos, and posterior work by Denis Sciama, Dicke, Schrödinger, and other authors, and also taking into account Dirac's Large Number Hypothesis, we get some expressions, such as:
$$
\frac{GM}{c^{3}\tau} \sim 1
$$
where $G$ is the universal gravitational constant, $M$ the mass of the universe, $c$ the speed of light and $\tau$ the age of the universe, and taking into account:
$$
G = c^{2} \sum_{i} \frac{r_{i}}{m_{i}} = c^{2} \frac{R}{M}
$$
where $R$ is the radius of the universe, we get:
$$
\frac{R}{c \tau} \sim 1
$$
Regardless of the validity of such, what's interesting is the definition of "laws" and "constants" in a purely relational and impredicative manner, to the extent that the unity might represent was being presented previously as the organization $\Phi$. Such would be refered to as meta-dynamics. Such systems, eventhough thermodynamically open, as is the case of organisms, and which follow very clearly physical "law", are organizationally or operationally closed.\\
$\Phi$ would be the "universal" property of such system. There would be an analogue of inertia, at the operational or organizational level. Unless there are pertubations to such system, no process or relations between them are going to change. If a pertubation happens, all other processes (and respective relations) collectively compensate such pertubation (potentially originating $n$-th order pertubations, which would be compensated from within the system). These processes evolve in order to conserve $\Phi$. Any process or relation is defined from within taking into account all other processes "composing" $\Phi$.\\
What needs to be understood is how such processes in this type of system behave. Not how they look from "outside" (following physical law), but how they look from "within" (meta-level). Any process in such system would behave in complete relation to all others.\\
This is a hard problem in analogy, on finding analogous meta-dynamics for such organization. Lest we forget the obvious, that all our measurements and models are relational. All of them take into account measurements of the world with relation to something else. What we see in the world is a reflection of our constraints, and no small world (our models and formalization) or combinations of them will ever exhaust the large and open-ended world in front of us.\\ 
\end{document}
